% Author: Adolfo Centeno 
% Kubeet Corp
% www.kubeet.com

 
\documentclass{beamer}
\setbeamertemplate{navigation symbols}{}
\usepackage[utf8]{inputenc}
\usepackage{beamerthemeshadow}
\usepackage{listings}

\begin{document}
\title{Goggle Compute Engine install - Motum}  
\author{Adolfo Centeno}
\date{\today} 

\begin{frame}
\titlepage
\end{frame}

\begin{frame}\frametitle{Table of contents}\tableofcontents
\end{frame} 


\section{Install, Google Compute Engine Client} 
\begin{frame}\frametitle{The cloud computing stack – SaaS, PaaS, and IaaS} 

We can represent cloud computing as a stack of three different categories:
\begin{itemize}
\item Software as a Service (SaaS)   
\item Platform as a Service (PaaS) 
\item Infrastructure as a Service (IaaS)
 

\end{itemize} 



\end{frame}

\subsection{Google App Engine (GAE)}
\begin{frame} 

Hosting + Compute 

There are two options if we want to host an application on Google Cloud Platform:



\begin{enumerate}

\item
Google App Engine: This is Google's PaaS and it will not be covered in this project.

\item
Google Compute Engine: This is Google's IaaS and lets users run virtual
machines on Google's infrastructure with a variety of hardware and
software configurations.
\end{enumerate}

\end{frame}


\subsection{Instalación de GCE Client (I)}


\defverbatim[colored]\lstA{
 \begin{lstlisting}[language=bash,showstringspaces=false, basicstyle={\tiny}, keywordstyle=\color{red}]
export CLOUD_SDK_REPO="cloud-sdk-$(lsb_release -c -s)"
 \end{lstlisting}
}


\defverbatim[colored]\lstB{
 \begin{lstlisting}[language=bash,showstringspaces=false, basicstyle={\tiny}, keywordstyle=\color{red}]
echo "deb http://packages.cloud.google.com/apt $CLOUD_SDK_REPO main" | 
     sudo tee -a /etc/apt/sources.list.d/google-cloud-sdk.list
 \end{lstlisting}
}

\defverbatim[colored]\lstC{
 \begin{lstlisting}[language=bash,showstringspaces=false, basicstyle={\tiny}, keywordstyle=\color{red}]
curl https://packages.cloud.google.com/apt/doc/apt-key.gpg | sudo apt-key add -
 \end{lstlisting}
}


\begin{frame} 

Windows install

\begin{block}{Update and install the Cloud SDK in windows}
\url{https://cloud.google.com/sdk/downloads}
\end{block}

	
\end{frame}


\begin{frame} 

Ubuntu/Debian install (Part I)

\begin{block}{Create an environment variable for the correct distribution}
\lstA
\end{block}


\begin{block}{Add the Cloud SDK distribution URI as a package source:}
\lstB
\end{block}

\begin{block}{Import the Google Cloud public key:}
\lstC
\end{block}


	
\end{frame}


\subsection{Instalación de GAE (II)}

\defverbatim[colored]\lstD{
 \begin{lstlisting}[language=bash,showstringspaces=false, basicstyle={\tiny}, keywordstyle=\color{red}]
sudo apt-get update && sudo apt-get install google-cloud-sdk
 \end{lstlisting}
}


\defverbatim[colored]\lstE{
 \begin{lstlisting}[language=bash,showstringspaces=false, basicstyle={\tiny}, keywordstyle=\color{red}]
sudo apt-get install google-cloud-sdk-app-engine-python
 \end{lstlisting}
}

\defverbatim[colored]\lstauthlogin{
 \begin{lstlisting}[language=bash,showstringspaces=false, basicstyle={\tiny}, keywordstyle=\color{red}]

$ sudo gcloud auth login  # login to Google 

NOTE: 
gmail user : adolfo.centeno@kubeet.com
password   : 5i5i5i5i

 \end{lstlisting}
}


\defverbatim[colored]\lstconnect{
 \begin{lstlisting}[language=bash,showstringspaces=false, basicstyle={\tiny}, keywordstyle=\color{red}]
# set the default project 
$ sudo gcloud config set project sanguine-office-187416  

# print virtual private servers
$ sudo gcloud compute instances list  

# check network connectivity
$ ping                  

# login with secure shell (ssh) to compute instance with [username]
$ sudo gcloud compute ssh [username]@jenkins --zone us-east1-b

# show home directory
$ ls /home 

 \end{lstlisting}
}



\defverbatim[colored]\lstfirewall{
 \begin{lstlisting}[language=bash,showstringspaces=false, basicstyle={\tiny}, keywordstyle=\color{red}]
# show firewall list
$ sudo gcloud compute firewall-rules list  

# create new rule to open 8085 port
$ sudo gcloud compute firewall-rules create http_8085 allow -tcp:8085  


 \end{lstlisting}
}

\begin{frame} 

Ubuntu/Debian install  (Part II)

\begin{block}{Update and install the Cloud SDK:}
\lstD
\end{block}


\begin{block}{Install the additional component:}
\lstE
\end{block}



	
\end{frame}




\begin{frame} 

GCE/login

\begin{block}{Run gcloud auth login to get started:}
\lstauthlogin
\end{block}
	
\end{frame}



\begin{frame} 

Connect to Google Compute Engine

\begin{block}{Set project/connect with ssh to GCE}
\lstconnect
\end{block}

	
\end{frame}



\section{GCE firewall} 
\begin{frame}\frametitle{GCE firewall} 

\begin{block}{Open firewall in GCE}
\lstfirewall
\end{block}


\end{frame}

\end{document}