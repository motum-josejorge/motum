% Author: Adolfo Centeno 
% Kubeet Corp
% www.kubeet.com

 
\documentclass{beamer}
\setbeamertemplate{navigation symbols}{}
\usepackage[utf8]{inputenc}
\usepackage{beamerthemeshadow}
\usepackage{listings}
\usepackage{hyperref}


\begin{document}
\title{postgres install - motum}  
\author{Adolfo Centeno Tellez}
\date{\today} 

\begin{frame}
\titlepage
\end{frame}

\begin{frame}\frametitle{Table of contents}\tableofcontents
\end{frame} 




\section{Conceptos} 
\begin{frame}\frametitle{postgres} 

 Conceptos:
 
\begin{itemize}
\item install
\item connect/createdb    
\item create tables
\item insert some values
\item select values..
 

\end{itemize} 



\end{frame}

\subsection{install}

\begin{frame} 
Install from here..

\url{https://www.digitalocean.com/community/tutorials/como-instalar-y-utilizar-postgresql-en-ubuntu-16-04-es}


\end{frame}


\subsection{createdb / connect}

\defverbatim[colored]\lstcreatedb{
 \begin{lstlisting}[language=html,showstringspaces=false, basicstyle={\tiny}, keywordstyle=\color{red}]
 
 $ sudo -u postgres createdb shopcar
 $ sudo -u postgres createuser adsoft
 $ sudo -u postgres psql
 postgres=# alter user adsoft with encrypted password '5i5i5i5i';
 postgres=# grant all privileges on database shopcar to adsoft;
 
 \end{lstlisting}
}

\defverbatim[colored]\lstconnect{
 \begin{lstlisting}[language=html,showstringspaces=false, basicstyle={\tiny}, keywordstyle=\color{red}]

$ psql -d shopcar -U adsoft	
shopcar=# \list	
shopcar=# \conninfo 
 

 \end{lstlisting}
}

\begin{frame}


\begin{block}{create database}
  \lstcreatedb
\end{block}

\begin{block}{connect to postgres}
  \lstconnect
  
\end{block}


\end{frame}


\subsection{create tables}

\defverbatim[colored]\lstcreateprod{
 \begin{lstlisting}[language=html,showstringspaces=false, basicstyle={\tiny}, keywordstyle=\color{red}]
  shopcar=#  create table productos (id SERIAL NOT NULL, 
  codigo int, 
  nombre  varchar(100), 
  precio float, 
  exist int,
  "createdAt" date not null default CURRENT_DATE,
  "updatedAt" date not null default CURRENT_DATE);

 \end{lstlisting}
}


\defverbatim[colored]\lstprodtest{
 \begin{lstlisting}[language=html,showstringspaces=false, basicstyle={\tiny}, keywordstyle=\color{red}]
 shopcar=# insert into productos(codigo, nombre, precio, exist) values(1, 
           'dron with raspberry pi 3', 8000, 10);
 shopcar=# select * from productos;
 \end{lstlisting}
}


\defverbatim[colored]\lstprodpk{
 \begin{lstlisting}[language=html,showstringspaces=false, basicstyle={\tiny}, keywordstyle=\color{red}]
 shopcar=# alter table productos add constraint pk_productos
           primary key (codigo);
 \end{lstlisting}
}



\defverbatim[colored]\lstproddelete{
 \begin{lstlisting}[language=html,showstringspaces=false, basicstyle={\tiny}, keywordstyle=\color{red}] 
 shopcar=# delete from productos where codigo=4;
 shopcar=# delete from productos where precio > 5000 
           and exist > 10;
 shopcar=# delete from productos where precio > 5000 
           or nombre like '%dron%' 
 \end{lstlisting}
}



\defverbatim[colored]\lstprodupdate	{
 \begin{lstlisting}[language=html,showstringspaces=false, basicstyle={\tiny}, keywordstyle=\color{red}] 
 shopcar=# update productos set precio=1500 where codigo=3;
 shopcar=# update productos set precio=1000,
           exist=10, descripcion = 'intel edison' 
           where codigo=3; 
 \end{lstlisting}
}


\begin{frame} 


\begin{block}{create table productos}
  \lstcreateprod
\end{block}


\begin{block}{testing ...}
  \lstprodtest
\end{block}

\begin{block}{creating Primary Key (PK) }
  \lstprodpk
\end{block}


\end{frame}


\begin{frame} 

\begin{block}{delete .. }
  \lstproddelete
\end{block}


\begin{block}{update .. }
  \lstprodupdate
\end{block}


\end{frame}



\end{document}
