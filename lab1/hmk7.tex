\documentclass[11pt]{article}
\usepackage{amsmath,amssymb,amsthm}
\usepackage[utf8]{inputenc}
\usepackage{graphicx}
\usepackage[margin=1in]{geometry}
\usepackage{fancyhdr}
\usepackage{hyperref}
\setlength{\parindent}{0pt}
\setlength{\parskip}{5pt plus 1pt}
\setlength{\headheight}{13.6pt}
\newcommand\question[2]{\vspace{.25in}\hrule\textbf{#1: #2}\vspace{.5em}\hrule\vspace{.10in}}
\renewcommand\part[1]{\vspace{.10in}\textbf{(#1)}}
\newcommand\algorithm{\vspace{.10in}\textbf{Instrucciones: }}
\newcommand\correctness{\vspace{.10in}\textbf{Valor: }}
\pagestyle{fancyplain}
\lhead{\textbf{\NAME\ (\ADSOFTID)}}
\chead{\textbf{HW\HWNUM}}
\rhead{LabWeb2018, \today}
\begin{document}\raggedright

\newcommand\NAME{Adolfo Centeno}  
\newcommand\ADSOFTID{adsoft}     
\newcommand\HWNUM{7}              

\question{1}{Lecturas} 

\part{a} \algorithm  Leer slides de jenkis de la carpeta doc, y branching models \url{http://nvie.com/posts/a-successful-git-branching-model/}

\correctness 20 pts.


\question{2}{Tutorial de superheroes} 

\part{a} \algorithm  Realizar la practica de la pagina de \url{https://angular.io}
\correctness 40 pts.

\question{3}{Instalar jenkins} 

\part{a} \algorithm  Instalar jenkins en su VPS \url{https://jenkins.io/doc/book/installing/}
\correctness 40 pts.


\end{document}
